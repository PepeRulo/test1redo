\documentclass{article}

\usepackage{pgfplots}
\usepackage[margin=0.75in, paperwidth=8.5in, paperheight=11in]{geometry}
\usepackage{setspace}
\usepackage{booktabs}
\usepackage[]{units}


% For nicely typeset tabular material
\usepackage{booktabs}

%%
% For graphics / images
\usepackage{graphicx}
\setkeys{Gin}{width=\linewidth,totalheight=\textheight,keepaspectratio}
\graphicspath{{graphics/}}

%%
% Additional
\usepackage{units}
\usepackage{amsmath,amsfonts,amsthm} % Math packages
\usepackage{mathtools}% http://ctan.org/pkg/mathtools
%\usepackage{mparhack}
\usepackage{sectsty} % Allows customizing section commands
%\usepackage[dvipsnames]{xcolor}
\usepackage{pgf,tikz}
%\usepackage{pgfplots}
%\usetikzlibrary{shapes,arrows}
%\usetikzlibrary{patterns,fadings}
%\usetikzlibrary{arrows}
% \usetikzlibrary{decorations.pathreplacing}
% \usetikzlibrary{snakes}
 %\usetikzlibrary{spy}
 %\usepackage{setspace}
% \usepackage{3dplot}
% \usepackage{cancel}
%\usepackage{physymb}
%\usepackage{braket}
%\usepackage{verbatim}
%\usepackage[x11names]{xcolor}                     %Additional colors
%\usepackage{euler}  



% The fancyvrb package lets us customize the formatting of verbatim
% environments.  We use a slightly smaller font.
\usepackage{fancyvrb}
\fvset{fontsize=\normalsize}


\begin{document}
\pagenumbering{gobble}

\doublespacing
\textbf{IB Physics }                        %%%(class number and section) 
 \hfill                             %%%(date of test)
${\bf Name: } \underline{\hspace{2.5in}}$

\begin{centering}
\vspace{1cm}
\textbf{Exam 1}\\
\end{centering}

\vspace{1cm}



$$\begin{tikzpicture}[scale=1]
     	
	
	
\draw [color=gray!80] (0,0)--(3,0)
   node [midway,anchor=south,inner sep=1pt, outer sep=1pt]{$r$};


 \begin{scope}[shift={(0,0)}, scale=1,rotate=120] 
 \draw[color=gray,dashed] (0,0) circle (3cm);
  \begin{scope}[shift={(0,-4)}, scale=0.75, rotate=-90] 
	  \draw[ thick,-stealth] (0,-0.1) -- (0,0.8) node [near start,anchor=east]{\scriptsize $\theta$};  
	  \draw[thick](-0.1,0) -- (0.1,0);
	   \draw[thick](0.2,-0.4) -- (0.2,-0.6);
	    \draw[ thick,-stealth] (0.1,-0.5) -- (1,-0.5) node [near start,anchor=north]{\scriptsize $r$};  
	  \end{scope}
 \fill[black] (0,0) circle (0.5mm) node [anchor = south] {$M$};  
 
 %\draw [very thick] (0,0) -- (0,-3);
 \fill[white] (0,-3) circle (0.5cm);
 \draw[very thick] (0,-3) circle (0.5cm);
  \draw[very thick] (0,0) circle (0.8cm);
 %\fill[color=white] (0.5,-2.5) -- (-0.5,-2.5) -- (-0.5,-3.5) -- (0.5,-3.5) -- cycle;
 % \draw[very thick] (0.5,-2.5) -- (-0.5,-2.5) -- (-0.5,-3.5) -- (0.5,-3.5) -- cycle;
  \draw (0,-3) node {$m$};
 \end{scope}
		  
    
   \end{tikzpicture}$$
   
   $$F_g=\frac{mMG}{r^2}$$
   \begin{table}[htbp]
\begin{center}
\footnotesize
\begin{tabular}{lllll}
\toprule
 Description              & Symbol          & Quantity                                                                \\
\midrule
  Gravitational Constant   & $G$             & $6.67 \times 10^{-11} \nicefrac{ \text{N}\cdot\text{m}^2}{\text{kg}^2}$ \\
    Mass of Earth   & $m_{earth}$           & $5.98\times 10^{24} \text{kg}$    \\
    Mass of Moon   & $m_{moon}$           & $7.36\times 10^{22} \text{kg}$    \\
    Radius of Earth     & $R_{earth}$           & $6.38 \times 10^{6} \text{m}$                   \\
     Radius of Moon    & $R_{moon}$           & $1.74 \times 10^{6} \text{m}$                   \\
     Orbital Radius of Earth    & $r_{earth}$           & $1.50 \times 10^{11} {\text{m}}$                   \\
     Orbital Radius of Moon    & $r_{moon}$           & $3.84 \times 10^{8} \text{m}$                   \\
      Period of Earth's Orbit     & $T_{earth}$ &   $365.24\text{ days}$                    \\
    Period of Moon's Orbit     & $T_{moon}$ &   $27.3\text{ days}$                    \\
                                     

\bottomrule
\end{tabular}
\end{center}
  \caption{A list of physical quantities.}
  \label{tab:font-sizes}
\end{table}

\newpage

The first question of the exam is worth 30 points.  The above table is required. 

 $\bf{1)}$ Consider the earth moving around the sun.
 
  \vspace{1cm}
 
 a.  Determine the orbital angular velocity of the earth.
 
 $$\omega=\frac{\Delta\theta}{\Delta t}$$
 $$\omega=\frac{2\pi}{T_{earth}}$$
 $$\omega=\frac{2\pi}{365.24*24*60*60}$$
 $$\omega=1.99 \times 10^{-7} \frac{\text{rad}}{\text{sec}}$$
    % $$\omega=1.99 \times 10^{-7} \frac{rad}{sec}$$ -text will be italicized as if it were composed of "variables", instead of units
 

 b.  Determine the speed of the earth relative to the sun.
 
 $$v=\frac{2\pi r}{T}$$ %2pi r is distance, T is time. 2pi/T is                          omega.
 $$v=\omega r_{earth}$$ %omega was found in the previous problem!
 $$v=1.99 \times 10^{-7} * 1.5 \times 10^{11}$$ %r value from                                                    table
 $$v=3.0 \times 10^{4} \frac{\text{m}}{\text{s}}$$ %rounded up                                                      from 29866
 
  
 c.  Determine centripetal acceleration of the earth relative to the sun.
 
 $$a_{cent}=\frac{v^{2}}{r_{earth}}$$
 $$a_{cent}=\frac{(3 \times 10^{4})^{2}}{1.5 \times 10^{11}}$$
    %v from the previous problem!
 $$a_{cent}=6.0 \times 10^{-3} \frac{\text{m}}{\text s^{2}}$$       %rounded up from 0.005947
 
 
 d. Determine the net force on the earth considering this acceleration.
 
 $$F_{net}=m_{earth}a$$ %"considering this acceleration" aka. the                        one found above
 $$F_{net}=5.98 \times 10^{24} * 6.0 \times 10^{-3}$$
 $$F_{net}=3.6 \times 10^{22} \text N$$ 
 
 
 \newpage
 e. Determine the mass of the sun from the above.
 
 $$M=\frac{F_{g}r^{2}}{mG}$$ %the force is the one found above, r                            is the orbital radius of earth and m                             is the mass of earth, G is the                                   universal gravitational constant
 $$M=\frac{(3.6 \times 10^{22}) (1.5 \times 10^{11})^{2}}{5.98       \times 10^{24} * 6.67 \times 10^{-11}}$$
 $$M=2.0 \times 10^{30} \text k \text g$$ 
 


  \newpage
  The second question is worth 30 points.  The table is required.
  
 $\bf{2)}$  Consider gravitation at the surface of the moon.
 
 \vspace{1cm}

  
 a.  Determine the acceleration due to gravity on the surface of the moon.
 
 \def\qeq{\mathrel{%
    \mathchoice{\QEQ}{\QEQ}{\scriptsize\QEQ}{\tiny\QEQ}%
}}
\def\QEQ{{%
    \setbox0\hbox{$m$}%
    \rlap{\hbox to \wd0{\hss/\hss}}\box0
}}

  $$F_{g}=\frac{ {\qeq} MG}{r^2}={\qeq}a$$
  
 $$F_{g}=\frac{M_{moon}G}{R^{2}_{moon}}=a$$
 $$F_{g}=\frac{(7.36 \times 10^{22}) (6.67 \times 10^{-11})}{(1.74    \times 10^{6})^{2}}=a$$
 $$a=1.62 \frac{\text{m}}{\text s^{2}}$$

 
 b.  Determine the launch velocity for circular orbit.
 
 $$a=a_{cent}=\frac{v^{2}}{R_{moon}}$$ %a_cent since it's moving                                        in a circle
 $$1.62=\frac{v^{2}}{R_{moon}}$$
 $$v=\sqrt{1.62 * 1.74 \times 10^{6}}$$
 $$v=1680 \frac{\text{m}}{\text{s}}$$
 
 
 c.  Determine the launch velocity for escape from the moon's gravity.
 
 $${E}=0$$
 $$KE + PE=0$$
 $$\frac{1}{2}{\qeq}v^{2} - \frac{{\qeq}M_{moon}G}{R_{moon}}=0$$
 $$v=\sqrt{\frac{2M_{moon}G}{R_{moon}}}$$
 $$v=\sqrt{\frac{2 * 7.36 \times 10^{22} * 6.67 \times              10^{-11}}{1.74 \times 10^{6}}}$$
 $$v=2370 \frac{\text{m}}{\text{s}}$$ %rounded from 2375.43
 
 
 \newpage
 d.  Determine the result of launching an object at 2000 m/s into the moon's horizon.
 
 $$\text {An elliptical orbit, since that velocity is in between the launch velocity and the escape velocity.}$$



  \newpage
Question three is worth 40 points.

  $\bf{3)}$  Consider a capacitor.  Two very large parallel conducting plates are connected to the leads of a 9 Volt battery.
  \vspace{0.5cm}
  
  a.  Determine the separation between the plates to generate a $30.0\  \frac{\text{N}}{\text{C}}$ electric field.
  
  $$E=\frac{-\Delta V}{x}$$ %definition
  $$x=\frac{-\Delta V}{E}$$ %variable to find
  $$x=\frac{9}{30}$$ %plug in numbers
  $$x=0.3 \text{m}$$
  
  
  b.  Determine the force of this electric field on a 0.012 Coulomb charge.
  
  $$F=qE$$ %because I'm not finding the electric force of one               point charge interacting with another.
  $$F=(0.012)(30)$$
  $$F=0.36 \text{N}$$
  
  
  c.  Determine the change in potential energy for the 0.012 C charge moving from the 9V plate to the 0V plate.
  
  $$PE=qV$$
  $$PE_{9V}=qV=(0.012)(9)=0.108$$
  $$PE_{0V}=qV=(0.012)(0)=0$$
  $$\Delta PE=-0.108 \text{ Joules}$$
  
  
  \newpage
  d. Draw the parallel plates and the electric field between them.  
  \begin{figure}[h]
  \centering
  \includegraphics[width=0.4\textwidth]{efield}
  \caption{Electric field between 0V and 9V parallel plates}
  \end{figure}
  
  
  
\end{document}